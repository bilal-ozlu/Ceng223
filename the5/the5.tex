\documentclass[12pt]{article}
\usepackage[utf8]{inputenc}
\usepackage{float}
\usepackage{amsmath}


\usepackage[hmargin=3cm,vmargin=6.0cm]{geometry}
%\topmargin=0cm
\topmargin=-2cm
\addtolength{\textheight}{6.5cm}
\addtolength{\textwidth}{2.0cm}
%\setlength{\leftmargin}{-5cm}
\setlength{\oddsidemargin}{0.0cm}
\setlength{\evensidemargin}{0.0cm}

%misc libraries goes here


\begin{document}

\section*{Student Information } 
%Write your full name and id number between the colon and newline
%Put one empty space character after colon and before newline
Full Name : Bilal Özlü \\
Id Number :  1942614 \\

% Write your answers below the section tags
\section*{Answer 1}
 By Induction \\
 \\
 Basis: k=2 , $2^k$ = 4 vertices. A Hamilton cycle can be clearly seen [00, 01, 11, 10] \\
 \\
 Inductive: We must show how to produce $A_{k+1}$ graph from $A_k$ graph. \\
 Assume that $k>1$, k $\in$ $Z^+$ , and $A_k$ has a Hamilton cycle. \\
 Let [$c_1$, $c_2$,...$c_x$] are the binary digit vertices for x = $2^k$ \\
 Then [0$c_1$, 0$c_2$,...$0c_x$,$1c_1$, $1c_2$,...$1c_x$] are the binary digit vertices for 2x = $2^{k+1}$ ,because $2^{k+1}$ = $2^k$$\cdot$2 and we can produce it by adding each vertex '0' and '1'. \\
 So, $A_{k+1}$ has a Hamilton cycle if $A_k$ has a Hamilton cycle. \\ 

\section*{Answer 2}
 The chromatic number of the graph is 4 \\
 It is at least 3, because of the triangle of [E,F,G] \\
 Let's say E='green' , F='red' , G='blue' . \\
 C is different than F and G, then C='green'. \\
 A is different than G and C, then A='red'. \\
 H is different than F and C, then H='blue'. \\
 D is different than A,C,F; then D='blue'. \\
 I is different than D,E,F; then I must take a new colour. \\
 Because D='blue', E='green', F='red', we cannot color I with these three ones, we need a new colour. Let's say I='purple'. \\
 Note that chromatic number increased to 4. \\
 Lastly, B is different than A and I. B can be either green or blue, so it doesn't change the number. \\
 Hence, the chromatic number of the graph is exactly 4 [green,red,blue,purple]. \\ 

\section*{Answer 3}
 For $T_1$, The number of nodes for depths are: \\
 at height 0: 1 \\
 at height 1: 1! \\
 at height 2: 2$\cdot$1=2!  \\
 at height 3: 3$\cdot$2$\cdot$1=3! \\
 at height h: h$\cdot$(h-1)$\cdot$(h-2)...$\cdot$1=h! \\
 \\
 For $T_2$, The number of nodes for depths are: \\
 at height 0: 1 \\
 at height 1: h \\
 at height 2: h$\cdot$(h-1)  \\
 at height 3: h$\cdot$(h-1)$\cdot$(h-2) \\
 at height h: h$\cdot$(h-1)$\cdot$(h-2)...$\cdot$1=h! \\
 \\
 There is 1 node at height 0 in both $T_1$ and $T_2$ \\
 There are h! nodes at height h in both $T_1$ and $T_2$ \\
 \\
 For $h>n>0$, there are n! nodes at height n in $T_1$ \\
 For $h>n>0$, there are $\dfrac{h!}{(h-n)!}$ nodes at height n in $T_2$ \\
 $\dfrac{h!}{(h-n)!}$ is larger than (n!) as long as $h>n>0$ ; because $h>n$, $(h-1)$$>$$(n-1)$,...$(h-n+1)$$>$1 \\
 So, number of nodes in $T_1$ is smaller than number of nodes in $T_2$ at any height between 0 and h \\
 \\
 For h=0, number of nodes are equal (it's 1) \\
 For h=1, number of nodes are equal (it's 2) \\
 Otherwise(h$\geq$2), number of nodes in $T_1$ are smaller than number of nodes in $T_2$ \\
 Consequently, the number of nodes in $T_1$ is smaller or equal to the number of nodes in $T_2$. \\
 
\section*{Answer 4}
 If T has n nodes, it has (n-1) edges, because it is a tree. \\
 There are $\binom{n}{2}$ possible pairs for a graph having n nodes. \\
 Complement of a graph with (n-1) edges and n nodes, has $\binom{n}{2}$ - (n-1) edges. \\
 If the complement of T is also a tree, it gives $\binom{n}{2}$ - (n-1) = (n-1) \\
 So, $\binom{n}{2}$ = 2(n-1) \\
 $\dfrac{n(n-1)}{2}$ = 2(n-1) \\
 $\dfrac{n}{2}$ = 2 \\
 n=4 \\

\section*{Answer 5}


\section*{Answer 6}



\end{document}

​

