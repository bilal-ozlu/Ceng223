\documentclass[12pt]{article}
\usepackage[utf8]{inputenc}
\usepackage{float}
\usepackage{amsmath,amssymb}


\usepackage[hmargin=3cm,vmargin=6.0cm]{geometry}
%\topmargin=0cm
\topmargin=-2cm
\addtolength{\textheight}{6.5cm}
\addtolength{\textwidth}{2.0cm}
%\setlength{\leftmargin}{-5cm}
\setlength{\oddsidemargin}{0.0cm}
\setlength{\evensidemargin}{0.0cm}
%misc libraries goes here
\usepackage{tikz}

\begin{document}

\section*{Student Information } 
%Write your full name and id number between the colon and newline
%Put one empty space character after colon and before newline
Full Name : Bilal Özlü \\
Id Number : 1942614 \\

% Write your answers below the section tags
\section*{Answer 1}
 a) It is reflective. \\
 For (a,b), in this relation, for every 'a' there is a 'b' such  that a=b, because $\vert$a-b$\vert$=0$<$4 when a=b. \\
 \\
 b) It is symmetric. \\
 If $\vert$a-b$\vert$$<$4, then $\vert$b-a$\vert$$<$4. It means that if (a,b) exists, then (b,a) also exits. \\
 \\ 
 c) It is not transitive. \\
 Assume that the relation contains (a,b),(b,c) and (a,c) such that (a-b)=3, $\vert$3$\vert$$<$4  and (b-c)=2, $\vert$2$\vert$$<$4. So (a-c)=5, but $\vert$5$\vert$$\nless$4. Then (a,c) does not exist in the relation. And it is not transitive. \\
 
\section*{Answer 2}
 a) In order to be an equivalence relation, T must be reflexive, symmetric and transitive. \\
 It is reflexive. \\
 Given a$\in$S, we can clearly see that a/a=1 and it is rational, so aTa. \\
 \\
 It is symmetric. \\
 Assume a,b $\in$S are given and aTb. a/b is a non-zero raitonal number by definition, then it's inverse $\dfrac{1}{a/b}$=b/a is also a rational number. \\
 \\
 It is transitive. \\
 Assume a,b,c are given with aTb, bTc. By definition of T, a/b and b/c are rational numbers. Then, product of them (a/b)$\cdot$(b/c)= a/c is also rational, aTc. \\
 \\
 b) We can reach the distinct equivalence classes by providing a representative from each of them. \\
 Consider the subset A= $\lbrace$k-$\sqrt{5}$ : k is rational$\rbrace$ $\cup$ 1 of S. \\
 Elements of A represent all distinct equivalence classes of T. \\
 a=(x-y$\sqrt{5}$) $\in$S. If y=0, then a=x, so a/1=x is also rational and aT1. \\
 If y$\neq$0 then k=x/y. So we can see that  $\dfrac{a}{k-\sqrt{5}}$ = $\dfrac{x-y\sqrt{5}}{(x/y)-\sqrt{5}}$ = y, so aT(k-$\sqrt{5}$). \\
 Think that (k-$\sqrt{5}$) and (l-$\sqrt{5}$) are different elements of A. \\
 Let B = $\dfrac{k-\sqrt{5}}{l-\sqrt{5}}$ = ($\dfrac{k-\sqrt{5}}{l-\sqrt{5}}$)$\cdot$($\dfrac{l+\sqrt{5}}{l+\sqrt{5}}$) = $\dfrac{k \cdot l + \sqrt{5}(k-l)-5}{l^2-5}$ \\
 We can conclude this as $\sqrt{5}$ = $\dfrac{B \cdot (l^2-5) - k \cdot l + 5}{k-l}$ and it is rational which we know is not the case.\\
 Therefore, k-$\sqrt{5}$ and l-$\sqrt{5}$ are not equivalent. Also we can see that for every element k-$\sqrt{5}$ of A, k-$\sqrt{5}$/1 = k-$\sqrt{5}$ is also irrational and so k-$\sqrt{5}$ is not equivalent to 1. \\
 Hence, every two distinct elements of A are not equivalent and represent different classes. \\
 
\section*{Answer 3}
 Here is the adjaceny matrix for the listing elements of (1,2,3,4).
 R$_0$ = $\begin{bmatrix}
       0 & 1 & 0 & 0\\
       0 & 0 & 0 & 1\\
       0 & 0 & 0 & 0\\
       1 & 0 & 1 & 0\\
     \end{bmatrix}$\\
 For R$_1$ There is a new path from 1 to 4 going through (1, 2, 4) \\
 There is a new path from 1 to 3 going through (1, 2, 4, 3) \\
 There is a new path from 1 to 1 going through (1, 2, 4, 1) \\
 \\
 R$_1$ = $\begin{bmatrix}
       \underline{1} & 1 & \underline{1} & \underline{1} \\
       0 & 0 & 0 & 1\\
       0 & 0 & 0 & 0\\
       1 & 0 & 1 & 0\\
     \end{bmatrix}$\\
     \\
 For R$_2$ There is a new path from 2 to 1 going through (2, 4, 1) \\
 There is a new path from 2 to 3 going through (2, 4, 3) \\
 There is a new path from 2 to 2 going through (2, 4, 1, 2) \\
 R$_2$ = $\begin{bmatrix}
       1 & 1 & 1 & 1\\
       \underline{1} & \underline{1} & \underline{1} & 1\\
       0 & 0 & 0 & 0\\
       1 & 0 & 1 & 0\\
     \end{bmatrix}$\\
     \\
 For R$_3$ There is no new path from 3. Same. \\
 R$_3$ = $\begin{bmatrix}
       1 & 1 & 1 & 1\\
       1 & 1 & 1 & 1\\
       0 & 0 & 0 & 0\\
       1 & 0 & 1 & 0\\
     \end{bmatrix}$\\
     \\
 For R$_4$ There is a new path from 4 to 2 going through (4, 1, 2) \\
 There is a new path from 4 to 4 going through (4, 1, 2, 4) \\
 R$_4$ = $\begin{bmatrix}
       1 & 1 & 1 & 1\\
       1 & 1 & 1 & 1\\
       0 & 0 & 0 & 0\\
       1 & \underline{1} & 1 & \underline{1}\\
     \end{bmatrix}$\\
 R$_4$ is the matrix of the transitive closure. \\
 
\section*{Answer 4}
 a) [54,96,120,144] \\
 b) [4,18] \\
 c) No. \\
 d) No. \\
 e) [36,72,144] \\
 f) 36 \\
 g) [4,8,16,24] \\
 h) 24 \\
 
\section*{Answer 5}
 Both have 6 nodes and 8 edges. \\
 Degrees of G is : [4,3,3,2,2,2,2] in order of (b,a,e,c,d,f)\\
 Degrees of H is : [4,3,3,2,2,2,2] in order of (q,m,n,o,p,r) \\
 So their degrees are same and they have the same matrices. \\
 That means they are isomorphic. \\
 
\section*{Answer 6}
 a) If they have the same degree for each node, the numbers of edges are equal. \\
 So, the number of edges are equal from V$_1$ to V$_2$ and from V$_2$ to V$_1$. \\
 Then, $\mid$ $V_1$ $\mid$ = $\mid$ $V_2$ $\mid$ \\
 
\end{document}

​

