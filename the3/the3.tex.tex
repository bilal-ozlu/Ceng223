\documentclass[12pt]{article}
\usepackage[utf8]{inputenc}
\usepackage{float}
\usepackage{amsmath}


\usepackage[hmargin=3cm,vmargin=6.0cm]{geometry}
%\topmargin=0cm
\topmargin=-2cm
\addtolength{\textheight}{6.5cm}
\addtolength{\textwidth}{2.0cm}
%\setlength{\leftmargin}{-5cm}
\setlength{\oddsidemargin}{0.0cm}
\setlength{\evensidemargin}{0.0cm}

%misc libraries goes here


\begin{document}

\section*{Student Information } 
%Write your full name and id number between the colon and newline
%Put one empty space character after colon and before newline
Full Name : Bilal Özlü \\
Id Number : 1942614 \\

% Write your answers below the section tags
\section*{Answer 1}
 Let P(k) be the proposition that ($C^{k}$-1) is divisible by (C-1) for k$\geq$1 \\
 Basis: P(1) is true, because (C-1) is divisible by (C-1) \\
 Inductive: Assume that P(k) is true for an arbitrary integer x with x$\geq$1 and ($C^{x}$-1) is divisible by (C-1). Also P(k+1) must be true and ($C^{x+1}$-1) must be divisible by (C-1). \\
 ($C^{x+1}$-1) = $C^{x+1}$ - $C^{x}$  + $C^{x}$ -1 = ($C^{x}$(C-1)) + ($C^{x}$-1) \\
 ($C^{x}$(C-1)) is clearly divisible by (C-1) and ($C^{x}$-1) is also divisible if P(k) is true . Then, if (($C^{x}$-1) is divisible by (C-1), ($C^{x+1}$-1) is also divisible by (C-1). \\
 That means P(k+1) is true when P(k) is true. \\
\section*{Answer 2}
 Let P(n) be the proposition that (1-$\dfrac{1}{1+2}$)$\cdot$(1-$\dfrac{1}{1+2+3}$)$\cdot$...$\cdot$(1-$\dfrac{1}{1+2+3+...+n}$) = $\dfrac{n+2}{3n}$ for n$\geq$2. \\
 Basis:P(2)=(1-$\dfrac{1}{1+2}$) = $\dfrac{2}{3}$ = $\dfrac{4}{6}$, it is true. \\
 Inductive: Assume that P(k) is true for an arbitrary integer k with k$\geq$2. Then P(k+1) must also be true. \\
 P(k) = (1-$\dfrac{1}{1+2}$)$\cdot$(1-$\dfrac{1}{1+2+3}$)$\cdot$...$\cdot$(1-$\dfrac{1}{1+2+3+...+k}$) = $\dfrac{k+2}{3k}$ \\
 P(k+1) = (1-$\dfrac{1}{1+2}$)$\cdot$(1-$\dfrac{1}{1+2+3}$)$\cdot$...$\cdot$(1-$\dfrac{1}{1+2+3+...+k+(k+1)}$) = P(k)$\cdot$(1-$\dfrac{1}{1+2+3+...+(k+1)}$) \\
 1-$\dfrac{1}{1+2+3+...+(k+1)}$ = 1- $\dfrac{2}{(k+1)(k+2)}$ = $\dfrac{k^{2}+3k}{k^{2}+3k+2}$ , because $\sum_{n=1}^{k+1}$ n = $\dfrac{(k+1)(k+2)}{2}$ \\
 P(k) = $\dfrac{k+2}{3k}$ \\
 P(k+1) = ($\dfrac{k+2}{3k}$)$\cdot$($\dfrac{k^{2}+3k}{k^{2}+3k+2}$) = $\dfrac{(k+2)(k+3)(k)}{3(k)(k+1)(k+2)}$ = $\dfrac{(k+3)}{3(k+1)}$ \\
 That is true. P(k+1) is true when P(k) is true. \\
\section*{Answer 3}
Without constraints we have C(12,4) ways to make a quadrilateral. \\
No three vertices can be collinear : C(7,1)$\cdot$C(5,3)$\cdot$3 \\
No four vertices can be collinear : C(5,4)$\cdot$3 \\
Number of quadrilaterals that can be formed : C(12,4) - C(7,1)$\cdot$C(5,3)$\cdot$3 - C(5,4)$\cdot$3 = 270  \\
\section*{Answer 4}
 If we subtract 1 from odd positive numbers and 2 from even positive numbers, we can convert them non-negative even integers. \\
 ($x_{1}$-1) + ($x_{2}$-2) + ($x_{3}$-1) + ($x_{4}$-2) + ($x_{5}$-1) =60 \\
 ($x_{1}$-1) = 2a \\
 ($x_{2}$-2) = 2b \\
 ($x_{3}$-1) = 2c \\
 ($x_{4}$-2) = 2d \\
 ($x_{5}$-1) = 2e \\
 a+b+c+d+e=30, then the formula is C(5+30-1, 30) = C(34,30) = C(34,4)=46376 \\
\section*{Answer 5}
 Let $a_{n}$ denote the number of ways to arrange these courses without three  consecutive elective courses. \\
 Initial case : $a_{1}$ = 2 , $a_{2}$ = 5 , $a_{3}$ = 4 \\
 Ending with a must course : (...,must,must) : any schedule of length (n-2) without three consecutive elective courses $\rightarrow$ ($a_{n-2}$) \\
 Ending with a free elective course : (...,must,must,free) $\rightarrow$ ($a_{n-3}$) \\
 : (...,must,must,free,free) $\rightarrow$ ($a_{n-4}$) \\
 : (...,must,must,technical,free) $\rightarrow$ ($a_{n-4}$) \\
 total $\rightarrow$ 2($a_{n-4}$) + ($a_{n-3}$) \\
 Ending with a technical elective course : (...,must,must,technical) $\rightarrow$ ($a_{n-3}$) \\
 : (...,must,must,free,technical) $\rightarrow$ ($a_{n-4}$) \\
 : (...,must,must,technical,technical) $\rightarrow$ ($a_{n-4}$) \\
 total $\rightarrow$ 2($a_{n-4}$) + ($a_{n-3}$) \\
 All the ways : ($a_{n-2}$) + 2($a_{n-3}$) + 4($a_{n-4}$) \\
 $a_{n}$ = ($a_{n-2}$) + 2($a_{n-3}$) + 4($a_{n-4}$) \\
\section*{Answer 6}


\end{document}

​

